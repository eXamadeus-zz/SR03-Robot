% LaTeX template for ECE 496 Reports
% Last updated 27 January 2014 by Ben Ujcich

% Copyright (C) 2014 Laura Clancy, Julian Coy Katelyn Fry,
%                    Gregory Stephens, and Ben Ujcich

%% CHANGE REPORT TITLE HERE
\newcommand{\reporttitle}{
2-Wheeled Segway Robot Design:
Final Report
}

%% HEADER/PREAMBLE INFORMATION

\documentclass[11pt]{report}
\usepackage[T1]{fontenc}
\usepackage[utf8]{inputenc}

% "The font should be 11pt Times New Roman"
\usepackage{mathptmx}               

% "The body of the paper should use 1" margins on all sides."
\usepackage[margin=1in]{geometry}

% "Pages must be numbered, starting with 1 on the first page in the body of the report.
% The cover page should not be numbered. 
% Page numbers should be in the bottom-right corner of the page."
\usepackage{fancyhdr}
\pagestyle{fancy}
\fancyhead{}
\fancyfoot{}
\renewcommand{\headrulewidth}{0pt}
\fancyfoot[R]{\thepage}

% Set up customized spacing
\usepackage{setspace}

% Allows for Trademark Symbols
\usepackage{textcomp}

% Remove spacing between items in lists
\usepackage{enumitem}

% Remove extra spacing between titles of sections and subsections
\usepackage{titlesec}
\titlespacing\section{0pt}{0pt plus 4pt minus 2pt}{0pt plus 2pt minus 2pt}
\titlespacing\subsection{0pt}{0pt plus 4pt minus 2pt}{0pt plus 2pt minus 2pt}
\titlespacing\subsubsection{0pt}{0pt plus 4pt minus 2pt}{0pt plus 2pt minus 2pt}

% Set up BibTeX integration using IEEE citation format
\usepackage{cite}
\bibliographystyle{ieeetr}
\usepackage{url}

% Set bibliography to have a section header rather than chapter header
\makeatletter
\renewenvironment{thebibliography}[1]
     {\section*{Works Cited}% <-- this line was changed from \chapter* to \section*
      \@mkboth{\MakeUppercase\bibname}{\MakeUppercase\bibname}%
      \list{\@biblabel{\@arabic\c@enumiv}}%
           {\settowidth\labelwidth{\@biblabel{#1}}%
            \leftmargin\labelwidth
            \advance\leftmargin\labelsep
            \@openbib@code
            \usecounter{enumiv}%
            \let\p@enumiv\@empty
            \renewcommand\theenumiv{\@arabic\c@enumiv}}%
      \sloppy
      \clubpenalty4000
      \@clubpenalty \clubpenalty
      \widowpenalty4000%
      \sfcode`\.\@m}
     {\def\@noitemerr
       {\@latex@warning{Empty `thebibliography' environment}}%
      \endlist}
\makeatother

% Set up math
\usepackage{amsmath}
\usepackage{amsfonts}
\usepackage{amssymb}

% Set up graphics
\usepackage{graphicx}
\usepackage{float}

% Set up tables
\usepackage{tabularx}
\usepackage{booktabs}

%% START OF DOCUMENT

\begin{document}

% "The main body of text should use 1.5 spacing"
\begin{spacing}{1.5}

% Suppress page numbering on first page
\thispagestyle{empty}

% Title
% "The title should be centered and written in approximately 22pt font."
\vspace*{72pt}
{
\huge
\begin{center}
    \reporttitle
\end{center}
}
\vspace{72pt}

% Team Number
% "The Team number should be centered and written several lines below the title and should use a
% similar size font as the title."
{
\huge
\begin{center}
  Team SR03
\end{center}
}

% Team Members
% "Directly below the team identifier, team members should be listed alphabetically by last name, one
% per line, in approximately 14pt font. The column of names should be approximately centered on
% the page, but the names within the column should be left justified (so they all start at the same
% horizontal position)."
{
\Large 
\begin{center}
  \begin{tabular}{l}
    Laura Clancy \\
    Julian Coy \\
    Katelyn Fry \\
    Gregory Stephens \\
    Ben Ujcich
  \end{tabular}
\end{center}
}

% New page and reset page numbering
\clearpage
\setcounter{page}{1}

%% START EDITS BELOW %%




\section*{Problem Statement} %(0.5 pages)

The desired goal of the senior design project was to build a functioning robot that could successfully balance and move on two wheels as directed by the user. This design was meant to be based on the commercial Segway robot but created on a much smaller scale. The design requirements of the project included: 1) the robot will be roughly 18 inches high and no more than 9 inches wide and 2) without the use of RC Servo motors, the miniature Segway robot must be able to move forwards, backwards, and turn left and right. In our particular implementation, an individual power amplifier was constructed and the entire device was run from a single DC voltage supplied to the system. The system was controlled with the Tiva C Series TM4C123G LaunchPad and a Raspberry Pi computer. Our team encountered many difficulties in accomplishing the goal of balancing as outlined throughout this report.

\section*{Specifications}
We were given the following specifications to meet minimum requirements for the project \cite{Requirements}:
\begin{itemize}[noitemsep,nolistsep]
    \item The robot should stand roughly 18 inches high.
    \item The real time component of the system must be implemented using the Tiva C Series TM4C123G LaunchPad.
    \item The power for the motors must be provided using your own selected electronics.
    \item The only power supplied to the system should come from a single DC voltage 
source.
    \item A remote control must be implemented for the robot.
    \item The robot should be robust to significant perturbations.
    \item The primary performance metric will be the ability of the robot to traverse reasonable flat ground robustly. 
\end{itemize}

We based our design off of the following design principles:
\begin{itemize}[noitemsep,nolistsep]
    \item \emph{Choose off-the-shelf parts} rather than self-made parts whenever possible.
    \item \emph{Reuse and expand on open-source software libraries} to avoid spending time writing code that duplicates functionality that already exists elsewhere (and is likely more robust).
    \item \emph{Keep the hardware simple} by using the least amount of hardware necessary for operation to avoid additional potential points of failure.
    \item \emph{Modularize systems and components}. Each component should do one thing and do it well.
\end{itemize}

\section*{Final Design}

    \subsection*{Subsystem Description}

        \paragraph{Power Electronics}
        
        \paragraph{Sensors}
        
        \paragraph{Battery Power System}
        
        \paragraph{Controller}
        
        \paragraph{Microcontrollers}
        
        \paragraph{Mechanical System}
        
        \paragraph{Communications}
        
        The communications subsystem that was to be implemented on the robot was the Emmoco EDB-BLE development board using Bluetooth 4.0 Low Energy (BLE) standard. The reason for using the board was that by having a separate development board for Bluetooth\textsuperscript{\textregistered} communications, we would be able to debug the wireless protocols seperately from the rest of the system in order to save on development time and costs. Additionally, the EDB-BLE development board provided the em-framework to allow for building control apps for moobile devices such as Android and iOS phones.
        

    \subsection*{Subsystem Discussion}
    
        \paragraph{Power Electronics}
        
        \paragraph{Sensors}
        
        \paragraph{Battery Power System}
        
        \paragraph{Controller}
        
        \paragraph{Microcontrollers}
        
        \paragraph{Mechanical System}
        
        \paragraph{Communications}
        
        The communications system was difficult to interface with due to the use of two frameworks (em-framework and the Tiva framework) integrating with Simulink. Some time was spent attempting to write custom blocks for the Simulink model but ultimately proved to be too time consuming. We determined that our efforts would be better spent on getting the robot to balance rather than getting the communications system working.

\section*{Cost Accounting}

    \subsection*{Development Costs}
    
    \subsection*{Out of Pocket Development Costs}
    
    \subsection*{Final Artifact Cost}
    
    \subsection*{Out of Pocket Final Artifact Cost}

\section*{Performance Characterization and Discussion}

    Comparing the testing plans as set in the preliminary report to what was accomplished, we were not able to carry out most of the tests due to the inability of the Segway robot to balance correctly. Individual component testing plans and results are outlined below:
    
        \paragraph{Balancing System} \emph{"To test the gyroscope, the robot will be moved to different angles while the output of the gyroscope is recorded."} We were able to correctly determine that the Raspberry Pi was reading the gyroscope data correctly and sending it to 8 of the GPIO pins in the Tiva as an unsigned 8-bit integer with precision of one degree.
        
        \paragraph{Controller} \emph{"However, the majority of testing that will take place for this subsystem will come from the process of tuning the gains to write the PID controller.... While the system is running, different scopes will be used to compare the actual position with the desired position.  This data will be continually collected as the gains are adjusted to reduce the error."} We were not able to tune the controller to any signficiant degree due to lack of feedback data from the Tiva. The Simulink software provided with the Tiva was not able to correctly work when attempting to access real-time data to use to plot graphs of the feedback. Had we known the difficulties in integrating the Tiva with Simulink, we would have used the xPC to tune the gains of the controller.
        
        \paragraph{Motors} \emph{"For the preliminary torque test, the motors will be connected to the wheels and sent signals to move at different speeds. The speed of the rotating wheel will be visually observed and the control speeds will be calibrated in order to determine what needs to be sent to the motors to achieve minimum and maximum speeds."} The PWM signal was generated by the Tiva, and we tested that the PWM signal was correctly being sent from the board by using an oscilloscope. We tested the signal with duty cycles between 10 and 90 percent, and the motor's speed was responsive to the output.
        
        \paragraph{Power Amplifier and Voltage Regulator} \emph{"Both the power amplifier electronics and voltage regulator only need to be tested using a multimeter. The input voltages will be tested to verify their compatibility with the circuit components. Each electronic device will then be wired into the system and its voltage and current outputs measured before connecting them to the entire system."} The voltage regulator worked correctly, and we experienced no problems in powering the Tiva board or the Raspberry Pi computer.
        
        \paragraph{Mechanical System} \emph{"The majority of the testing for this system will be to test for robustness and durability.  This testing is necessary to ensure that the more delicate parts of the overall system will be protected.  Experimentation will also be done with the length of the PVC pipe and the mass of the counterweight to determine the best configuration to ensure a safe, durable, and controllable system."} The mechanical system was designed in such a way as to balance the weight of the components symmetrically so as to not have a significantly uneven weight pattern present. Important components such as the Tiva board and the Raspberry Pi were placed on the top of the robot to prevent them from being damaged if the robot were to fall over; similarly, the gyroscopic sensor was placed on the pipe to both ensure that an accurate angle measurement could be determined and that it would not be damaged.
        

\section*{Project Postmortem}

    \subsection*{Technical Postmortem}
    
    \subsection*{Nontechnical Postmortem}

\section*{Highlights}



% Bibliography
\bibliography{citationsfile}{}

\clearpage

\section*{Appendix A: Cost Accounting Tables}

%% END EDITS HERE %%

\end{spacing}

\end{document}
